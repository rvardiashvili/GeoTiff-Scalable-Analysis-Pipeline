\chapter{Conclusion}
\label{ch:conclusion}

\section{Summary of Contributions}

This thesis has presented the design, implementation, and evaluation of the \textbf{GeoSpatial Inference Pipeline (GSIP)}. The core contributions are:

\begin{enumerate}
    \item \textbf{A Robust Framework:} We successfully engineered a modular, model-agnostic pipeline that abstracts the complexities of geospatial data handling from the machine learning logic. Through the Adapter Pattern, we demonstrated seamless switching between legacy ResNets and modern Foundation Models (Prithvi).
    \item \textbf{Algorithmic Validity:} We formalized the \textbf{Sinusoidal Overlap-Tile Strategy}, proving mathematically and validating empirically that it eliminates the grid artifacts inherent in naive patching. We showed that this method is superior to simple averaging, preserving $C^1$ continuity.
    \item \textbf{Democratized Scaling:} Through the \textbf{Zone of Responsibility} memory model, we proved that it is possible to process "infinite" geospatial datasets on finite, consumer-grade hardware. We demonstrated a stable memory footprint while achieving a 3.75x throughput increase over standard single-threaded baselines.
\end{enumerate}

\section{Final Verdict}

Deep Learning in Earth Observation has rapidly matured from small-scale experimentation on carefully curated patches to operational monitoring of continents. However, the software engineering infrastructure required to deploy these models has lagged behind the algorithmic research. GSIP represents a step towards closing this gap. It serves as the necessary "Civil Engineering" infrastructure that allows the "Architectural" breakthroughs of Foundation Models to be safely and reliably deployed in the real world. By treating the Neural Network as a standard signal processing operator within a well-defined tiling algebra, we move closer to a future where planetary-scale AI analysis is as routine as a database query.

\section{Future Work}

The natural evolution of this work lies in deeper integration with standard geospatial infrastructure.
\begin{itemize}
    \item \textbf{OGC API - Processes:} Wrapping GSIP as a compliant OGC web service would allow it to be consumed directly by GIS clients like QGIS, enabling "Click-to-Analyze" workflows for non-technical users.
    \item \textbf{Time-Series Inference:} Extending the "Chunking" logic to the temporal dimension ($T$) is essential. This would require defining "Temporal Halos" to allow 3D-Convolutional networks or LSTMs to operate on long time series without boundary effects at the start/end of the sequence.
    \item \textbf{In-Database Execution:} Collaborating with the Rasdaman team to implement the GSIP logic as a native C++ module within the array database kernel could unlock true exascale performance.
\end{itemize}